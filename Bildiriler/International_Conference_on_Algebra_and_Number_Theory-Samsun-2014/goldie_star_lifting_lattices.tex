% Copyright 2004 by Till Tantau <tantau@users.sourceforge.net>.
%
% In principle, this file can be redistributed and/or modified under
% the terms of the GNU Public License, version 2.
%
% However, this file is supposed to be a template to be modified
% for your own needs. For this reason, if you use this file as a
% template and not specifically distribute it as part of a another
% package/program, I grant the extra permission to freely copy and
% modify this file as you see fit and even to delete this copyright
% notice. 

\documentclass[handout]{beamer}  
\usepackage{amsmath}
\usepackage{amsfonts}
\usepackage{amsthm}
\usepackage{amssymb}
\usepackage[backend=bibtex,citestyle=authoryear,maxnames=8,bibstyle=authortitle]{biblatex}

%\usetheme{Copenhagen} 
\usetheme{Frankfurt}
\useoutertheme[subsection=false]{miniframes}

\usecolortheme{whale}
\usepackage[utf8]{inputenc}
\setbeamercovered{transparent}
\setbeamertemplate{theorems}[numbered]


% There are many different themes available for Beamer. A comprehensive
% list with examples is given here:
% http://deic.uab.es/~iblanes/beamer_gallery/index_by_theme.html
% You can uncomment the themes below if you would like to use a different
% one:
%\usetheme{AnnArbor}
%\usetheme{Antibes}
%\usetheme{Bergen}
%\usetheme{Berkeley}
%\usetheme{Berlin}
%\usetheme{Boadilla}
%\usetheme{boxes}
%\usetheme{CambridgeUS}
%\usetheme{Copenhagen}
%\usetheme{Darmstadt}
%\usetheme{default}
%\usetheme{Frankfurt}
%\usetheme{Goettingen}
%\usetheme{Hannover}
%\usetheme{Ilmenau}
%\usetheme{JuanLesPins}
%\usetheme{Luebeck}
%\usetheme{Madrid}
%\usetheme{Malmoe}
%\usetheme{Marburg}
%\usetheme{Montpellier}
%\usetheme{PaloAlto}
%\usetheme{Pittsburgh}
%\usetheme{Rochester}
%\usetheme{Singapore}
%\usetheme{Szeged}
%\usetheme{Warsaw}

\title{Goldie${}_*$-Lifting Lattices}

% A subtitle is optional and this may be deleted

\author{Celil Nebiyev\inst{1} \and Hasan Hüseyin Ökten\inst{2}}
% - Give the names in the same order as the appear in the paper.
% - Use the \inst{?} command only if the authors have different
%   affiliation.

\institute[Ondokuz] % (optional, but mostly needed)
{
  \inst{1}%
  Department of Elementary Education, Education Faculty \\
  Ondokuz Mayıs University \and
  \inst{2}%
  Computer Programming Program, Technical Sciences Vocational School \\
  Amasya University}
% - Use the \inst command only if there are several affiliations.
% - Keep it simple, no one is interested in your street address.

\date{International Conference on Algebra and Number Theory, 2014}
% - Either use conference name or its abbreviation.
% - Not really informative to the audience, more for people (including
%   yourself) who are reading the slides online

\subject{Theoretical Computer Science}
% This is only inserted into the PDF information catalog. Can be left
% out. 

% If you have a file called "university-logo-filename.xxx", where xxx
% is a graphic format that can be processed by latex or pdflatex,
% resp., then you can add a logo as follows:

% \pgfdeclareimage[height=0.5cm]{university-logo}{university-logo-filename}
% \logo{\pgfuseimage{university-logo}}

% Delete this, if you do not want the table of contents to pop up at
% the beginning of each subsection:
\AtBeginSection[]
{
  \begin{frame}<beamer>{Outline}
    \tableofcontents[currentsection]
  \end{frame}
}

% Let's get started
\begin{document}

\begin{frame}
  \titlepage
\end{frame}
\begin{frame}{Abstract}
In this work, we define Goldie${}_*$-lifting lattices and show some characterizations of these lattices. 
We prove that if $ L $ is Goldie${}_* $-lifting, then Goldie$ {}_* $-supplemented and $ \oplus $-supplemented. 
We also prove that if $ L $ is Goldie${}_* $-supplemented and strongly $ \oplus $-supplemented, then Goldie${}_* $-lifting.\\

\end{frame}

\begin{frame}{Outline}
  \tableofcontents
  % You might wish to add the option [pausesections]
\end{frame}

% Section and subsections will appear in the presentation overview
% and table of contents.
\section{Introduction}
\subsection{Basic Definitions}

\begin{frame}
Throughout this paper, $ L $ denotes a complete modular lattice with smallest element 0 and 
greatest element 1. A lattice we will mean a complete modular lattice.
\end{frame}


\begin{frame}
\begin{itemize}[<+->]
    \item
        In a lattice $ L $ , for $ a,b \in L $ the join is denoted as $ a \vee b $ and the meet is denoted $ a \wedge b $.
    \item
        In a lattice $ L $, an element $ m \in L $ is called $
        maximal $ in $ L $ if there is no element between $ m $ and
        $ 1 $
    \item
        An element $ a $ of $ L $
        called $ small $ in $ L $, if $ a \vee b \neq 1 $ holds for
        every $ b \neq 1 $. This is denoted by $ a \ll L $. 
    \item
        An element $ c $ of $ L $ is called a $ supplement $ of $ b
        $ in $ L $ 
        if it is minimal relative to the property $ b \vee c = 1 $.
        Equivalently, an element $ c $ is a supplement of $ b $ in $
        L $ if and only if $ b \vee c = 1 $ and 
        $ b \wedge c \ll c/0 $
    \item
        An element $ c $ of $ L $ is called a $ weak \ supplement $
        of $ b $ in $ L $ if $ b \vee c = 1 $ and $ b \wedge c \ll L
        $. 
    \item
        A lattice $ L $ is called 
        $ supplemented $ (respectively, $ weakly \ supplemented $)
        if each element of $ L $ has a supplement (respectively,
        weak supplement) in $ L $
\end{itemize}
\end{frame}

\begin{frame}
\begin{itemize}[<+->]
    \item
        For $ a \in L $, we said that $ b \in L $ $ complement $ of $ a $ if $ a \wedge b = 0 $ and $ a \vee b = 1 $.
        It is denoted by $ a \oplus b = 1 $. $ a $ and $ b $ called direct summands of $ L $. A lattice $ L $ is called $ complemented $ if each
        element in $ L $ has at least one complemented.
    \item
        An element $ a $ of $ L $ has $ ample \  supplements $ in $ L $ if for every $ t \in L $ with  $ a \vee t = 1 $, 
        there is a supplement $ u $ of $ a $ with $ u \leq t $. 
        $ L $ is called $ amply$ $ supplemented $ if all elements of $ L $ have ample supplements in $ L $.
    \item
        For $ a,b \in L $ such that $ a \leq b $, we said that $ b \ lies \ above \ a $ if $ b \ll 1/a $.
    \item
        In a lattice $ L $, the meet of all maximal (except from 1) elements in $ L $ is called $ radical $ of $ L $, denoted $rad(L) $. 
    \item 
        $ m $ is a $Rad$-$supplement $ of $ n $ in $ L $, if $  m \vee n = 1 $ and $ m \wedge n \leq Rad(m/0) $.
        A lattice $ L $ is called $ Rad $-$supplemented $ if every element of $ L $ has a $ Rad $-$ supplement $
    \item
        A lattice $ L $ is called $ distributive $ if for any elements 
        $ a,b,c $ of $ L $, $ a \wedge ( b \vee c ) = ( a \wedge b ) \vee ( a \wedge c) $ holds.
\end{itemize}
\end{frame}
Let ¢



%%%%%%%%%%%%% Beta Yildiz Bagintisi %%%%%%%%%%%%%%%%
\section{$ \beta_* $ Relation on Lattices}
\subsection*{}
\begin{frame}
\setbeamercovered{invisible}

\begin{definition}
 Let $ a,b $ be elements of $ L $. We define a relation $ \beta_* $ on the elements of $ L $ by 
  $ a \beta_* b $ if and only if for each $ t \in L $ such that $ a \vee t = 1 $ then $ b \vee t = 1 $ and for each 
  $ k \in L $ such that $ b \vee k = 1 $ then $ a \vee k = 1 $. 
\end{definition}
\pause
Following theorem give us equivalent conditions of that definition.
\pause

    \begin{theorem} \label{4.1.3}
      Let $ a,b $ be elements of $ L $. Then,
      \begin{enumerate}[(i)]
    
        \item
          $ a \beta_* b $ if and only if $ a \vee c = 1 $ and $ b \vee c = 1 $ for each $ c \in L $ 
          such that $ a \vee b  \vee c = 1 $.  \label{4.1.3.1}
    
        \item
          $ a \beta_* b $ if and only if $ a \vee b \ll 1/a $ and $ a \vee b \ll 1/b $.  \label{4.1.3.2}
    
      \end{enumerate}
    \end{theorem}
\end{frame}

%%%%%%%%%%%%%%%%%%%%%%%%%%%%%%%%%%%%%%%%%%%%%%%%%%%%%%%%%%%%%%%%%%%%%
%%%%%%%%%%%%%%%%%%%%%%%%%%%%%%%%%%%%%%%%%%%%%%%%%%%%%%%%%%%%%%%%%%%%%
%%%%%%%%%%%%%%%%%%%%%%%%%%%%%%%%%%%%%%%%%%%%%%%%%%%%%%%%%%%%%%%%%%%%%
\begin{frame}
 \begin{theorem}\label{4.1.8}
  Let $ a,b $ be elements of $ L $ such that $ a \leq b $. 
  If $ b $ lies above $ a $ then, $ a \beta_* b $. 
\end{theorem}

\begingroup
\setbeamercolor{block title}{bg=green!30!black,fg=white}
\begin{proof}
  Assume $ b $ lies above $ a $. Then, $ b \in 1/a $. Since $ a \leq b $ for any $ t \in L $ such that $ a \vee t = 1 $, $ b \vee t = 1 $. 
  Conversely, let $ k \in L $ with $ b \vee k = 1 $. Then $ b \vee a \vee k = 1 $. Since $ b \in 1/a $ and $ a \vee k \in 1/a $, $ a \vee k = 1 $. Hence $ a \beta_* b $.
\end{proof}
\endgroup
\end{frame}


%%%%%%%%%%%%%%%%%%%%%%%%%%%%%%%%%%%%%%%%%%%%%%%%%%%%%%%%%%%%%%%%%%%%%
%%%%%%%%%%%%%%%%%%%%%%%%%%%%%%%%%%%%%%%%%%%%%%%%%%%%%%%%%%%%%%%%%%%%%
%%%%%%%%%%%%%%%%%%%%%%%%%%%%%%%%%%%%%%%%%%%%%%%%%%%%%%%%%%%%%%%%%%%%%
\begin{frame}
\begin{lemma}\label{4.1.9}
  Let $ a,b,c $ be elements of $ L $.
  If $ a \vee b = 1 $ and $ ( a \wedge b ) \vee c =1 $
  then $ a \vee(b \wedge c)=b \vee ( a \wedge c ) = 1 $.
\end{lemma}
\begingroup
\setbeamercolor{block title}{bg=green!30!black,fg=white}
\begin{proof}
  Assume $ a \vee b = 1 $ and $ ( a \wedge b ) \vee c = 1 $. Since $ ( a \wedge b ) \vee c = 1 $, $ a = a \wedge 1 = a \wedge \left[ ( a\wedge b ) \vee c \right]= ( a \wedge b ) \vee ( a \wedge c ) $. Then $ 1 = a\vee b $, 
  $ ( a \wedge b ) \vee ( a \wedge c ) \vee b = b \vee ( a \wedge c )$. Similarly $ a \vee ( b \wedge c ) = 1 $.
\end{proof}
\endgroup
\end{frame}

%%%%%%%%%%%%%%%%%%%%%%%%%%%%%%%%%%%%%%%%%%%%%%%%%%%%%%%%%%%%%%%%%%%%%
%%%%%%%%%%%%%%%%%%%%%%%%%%%%%%%%%%%%%%%%%%%%%%%%%%%%%%%%%%%%%%%%%%%%%
%%%%%%%%%%%%%%%%%%%%%%%%%%%%%%%%%%%%%%%%%%%%%%%%%%%%%%%%%%%%%%%%%%%%%
\begin{frame}
\setbeamercovered{invisible}
The following theorem show us the relationship between the $ \beta_* $ relation and supplements.
\pause
\begin{theorem}\label{4.1.10}
  Let $ a,b \in L $. If $ a\beta_* b $ then the following conditions hold.
  \begin{enumerate}[(i)]
    \item
      If there are exists supplements of $ a $ and $ b $ then these are same. \label{4.1.10.1}
    \item
      If there are exists weak supplements of $ a $ and $ b $ then these are same. \label{4.1.10.2}
  \end{enumerate}
\end{theorem}
\end{frame}


%%%%%%%%%%%%%%%%%%%%%%%%%%%%%%%%%%%%%%%%%%%%%%%%%%%%%%%%%%%%%%%%%%%%%
%%%%%%%%%%%%%%%%%%%%%%%%%%%%%%%%%%%%%%%%%%%%%%%%%%%%%%%%%%%%%%%%%%%%%
%%%%%%%%%%%%%%%%%%%%%%%%%%%%%%%%%%%%%%%%%%%%%%%%%%%%%%%%%%%%%%%%%%%%%
\begin{frame}
\begin{corollary}\label{4.1.12}
  Let $ x,y,c \in L $ such that $ x \leq y $ and $ c $ is a weak supplement of $ x $ in $ L $. Then 
  $ x \beta_* y $ if and only if $ y \wedge c \ll L $.
\end{corollary}

\begingroup
\setbeamercolor{block title}{bg=green!30!black,fg=white}
\begin{proof}
  $ ( \Rightarrow ) $: 
  Clear from Theorem \ref{4.1.10} \ref{4.1.10.2}. \\
  $ ( \Leftarrow ) $:  
  Since $ x \leq y $, for any element $ t $ of $ L $ such that $ x \vee t = 1 $, $ y \vee t = 1 $. 
  Let $ k \in L $ such that $ y \vee k = 1 $. Since $ c $ is weak supplement of $ x $ in $ L $, 
  $ x \vee c = 1 $ and $ x \wedge c \ll L $. Therefore $ y \wedge ( x \vee c ) = 1 \wedge y $, and so 
  $ y = x \vee ( y \wedge c ) $. Hence $ k \vee x \vee ( y \wedge c ) = 1 $. Since $ y \wedge c \ll L $, 
  $ x \vee k = 1 $. Thus $ x \beta_* y $.
\end{proof}
\endgroup
\end{frame}


%%%%%%%%%%%%%%%%%%%%%%%%%%%%%%%%%%%%%%%%%%%%%%%%%%%%%%%%%%%%%%%%%%%%%
%%%%%%%%%%%%%%%%%%%%%%%%%%%%%%%%%%%%%%%%%%%%%%%%%%%%%%%%%%%%%%%%%%%%%
%%%%%%%%%%%%%%%%%%%%%%%%%%%%%%%%%%%%%%%%%%%%%%%%%%%%%%%%%%%%%%%%%%%%%
\begin{frame}
\begin{theorem}\label{4.1.19}
  Let $x,y,a,b \in L $. If $ a,b \ll L, x \leq y \vee b $ and $ y \leq x \vee a $ then $ x \beta_* y $.
\end{theorem}

\begingroup
\setbeamercolor{block title}{bg=green!30!black,fg=white}
\begin{proof}
  Let $ k \in L $ such that $ x \vee y \vee k = 1 $. Since $ x \leq y \vee b $, $ y \vee b \vee k = 1 $. 
  From $ b \ll L $, $ y \vee k = 1 $. Similarly, $ x \vee k = 1 $. 
  Hence, by Theorem \ref{4.1.3} (\ref{4.1.3.1}), $ x \beta_* y $.
\end{proof}
\endgroup
\end{frame}

%%%%%%%%%%%%%%%%%%%%%%%%%%%%%%%%%%%%%%%%%%%%%%%%%%%%%%%%%%%%%%%%%%%%%
%%%%%%%%%%%%%%%%%%%%%%%%%%%%%%%%%%%%%%%%%%%%%%%%%%%%%%%%%%%%%%%%%%%%%
%%%%%%%%%%%%%%%%%%%%%%%%%%%%%%%%%%%%%%%%%%%%%%%%%%%%%%%%%%%%%%%%%%%%%
\begin{frame}
\begin{theorem}\label{4.1.20}
  Let $ x_1,x_2,y_1,y_2 \in L $. If $ x_1 \beta_* y_1 $ and $ x_2 \beta_* y_2 $ then $ ( x_1 \vee x_2 ) \beta_* ( y_1 \vee y_2 ) $.
\end{theorem}
\begingroup
\setbeamercolor{block title}{bg=green!30!black,fg=white}
\begin{proof}
  Let $ k \in L $ such that $ (x_1 \vee x_2 ) \vee (y_1 \vee y_2 ) \vee k = 1 $. Since $ x_1 \beta_* x_2 $, 
  $ y_1 \vee x_2 \vee y_2 \vee k = 1 $ and $ x_1 \vee x_2 \vee y_2 \vee k = 1 $. Also since 
  $ x_2 \beta_* y_2 $, $ y_1 \vee y_2 \vee k = 1 $ and $ x_1 \vee x_2 \vee k = 1 $. 
  By Theorem \ref{4.1.3} (\ref{4.1.3.1}), $ ( x_1 \vee x_2 ) \beta_* ( y_1 \vee y_2 ) $.
\end{proof}
\endgroup
\begin{corollary}\label{4.1.21}
  Let $ x, y_1, y_2,...,y_n \in L $. If $ x \beta_* y_i $ for $ i=1,2,...,n $ then $ \displaystyle x \beta_* \bigvee_{i=1}^n y_i $.
\end{corollary}
\end{frame}


\section{$ G_* $-Lifting Lattices}
\subsection*{}

%%%%%%%%%%%%%%%%%%%%%%%%%%%%%%%%%%%%%%%%%%%%%%%%%%%%%%%%%%%%%%%%%%%%%
%%%%%%%%%%%%%%%%%%%%%%%%%%%%%%%%%%%%%%%%%%%%%%%%%%%%%%%%%%%%%%%%%%%%%
%%%%%%%%%%%%%%%%%%%%%%%%%%%%%%%%%%%%%%%%%%%%%%%%%%%%%%%%%%%%%%%%%%%%%
\begin{frame}
\begin{definition}
 Let $ L $ be a lattice. $ L $ is $ G_* $-$ lifting $ if, for every $ x \in L $ there is a direct summand $ d $ of $ L $ such that $ x \beta_* d $.
\end{definition}
\end{frame}
%%%%%%%%%%%%%%%%%%%%%%%%%%%%%%%%%%%%%%%%%%%%%%%%%%%%%%%%%%%%%%%%%%%%%
%%%%%%%%%%%%%%%%%%%%%%%%%%%%%%%%%%%%%%%%%%%%%%%%%%%%%%%%%%%%%%%%%%%%%
%%%%%%%%%%%%%%%%%%%%%%%%%%%%%%%%%%%%%%%%%%%%%%%%%%%%%%%%%%%%%%%%%%%%%
\begin{frame}
\setbeamercovered{invisible}
Following theorems give us some characterizations $ G_* $-$ lifting $ lattices.
\pause
\begin{theorem} \label{4.2.3}
$ L $ is $ G_* $-$ lifting $($ G_* $-$ supplemented $) if and only if for each $ x \in L $ there exist a direct summand $ d $ (supplement $ s $)
and a small element in $ L $ such that $ x \vee h = d \vee h = x \vee d $ ($ x \vee h = s \vee h = x \vee s $).
\end{theorem}
\begingroup
\setbeamercolor{block title}{bg=green!30!black,fg=white}
\begin{proof}
Assume that $ L $ is $ G_* $-$ lifting $. There exist a direct summand $ d $ such that $ x \beta_* d $. 
Hence there exist $ w \in L $ such that $ d \vee w = 1 $ and $ d \wedge w \ll d/0 $. 
By Theorem \ref{4.1.20}, $ x \beta_* (x \vee d) $ and $ d \beta_* (d \vee x) $.
From Theorem \ref{4.1.10} (\ref{4.1.10.2}), $ w $ is a weak supplement for $ d $, $ x $ and $ d \vee x $. 
By the Modular Law, $ x \vee h = d \vee h = x \vee d $ where $ h = (d \vee x) \wedge w \ll L $. 
The converse follows from Theorem \ref{4.1.19}
\end{proof}
\endgroup
\end{frame}


%%%%%%%%%%%%%%%%%%%%%%%%%%%%%%%%%%%%%%%%%%%%%%%%%%%%%%%%%%%%%%%%%%%%%
%%%%%%%%%%%%%%%%%%%%%%%%%%%%%%%%%%%%%%%%%%%%%%%%%%%%%%%%%%%%%%%%%%%%%
%%%%%%%%%%%%%%%%%%%%%%%%%%%%%%%%%%%%%%%%%%%%%%%%%%%%%%%%%%%%%%%%%%%%%
\begin{frame}
\fontsize{10pt}{7.5}\selectfont
\begin{corollary} \label{4.2.4}
    Let $ L $ be a lattice,
    \begin{enumerate}[(i)]
        \item 
            If for each $ x \in L $ there exist a direct summand $ s $ and a small element $ h $ in $ L $ 
            such that $ x = s \vee h $, then $ L $ is $ G_* $-$ lifting $. The converse holds if $ L $ is also distributive.
        \item
            Let $ L $ be $ G_* $-$ lifting $ and $ x \in L $ such that $ Rad(L) \leq x $. Then $ x = s \vee h $, where 
            $ s $ is a direct summand and $ h \ll L $. 
      \end{enumerate}
\end{corollary}
\begingroup
\setbeamercolor{block title}{bg=green!30!black,fg=white}
\begin{proof}
    \begin{enumerate}[(i)]
        \item 
            From Theorem \ref{4.1.19} the hypothesis implies that $ L $ is $ G_* $-$ lifting $. 
            Assume that $ L $ is $ G_* $-$ lifting $ an distributive. Let $ x \in L $. 
            Then there are $ s,t \in L $ such that $ x \beta_* s $, $ s \vee t = 1 $ and $ s \wedge t = 0 $. 
            Then $ x \vee t = 1 $. So $ s = s \wedge (x \vee t) = (s \wedge x) \vee (s \wedge t) = s \wedge x $. 
            Hence $ s \leq x $. From Theorem \ref{4.1.12}, $ t $ is a weak supplement of $ x $, so $ x \wedge t \ll L $. 
            Thus $ x = x \wedge (s \vee t) = s \vee h $, where $ h = x \wedge t $.
        \item
            This part follows from Theorem \ref{4.2.3}.
    \end{enumerate}
\end{proof}
\endgroup
\end{frame}


%%%%%%%%%%%%%%%%%%%%%%%%%%%%%%%%%%%%%%%%%%%%%%%%%%%%%%%%%%%%%%%%%%%%%
%%%%%%%%%%%%%%%%%%%%%%%%%%%%%%%%%%%%%%%%%%%%%%%%%%%%%%%%%%%%%%%%%%%%%
%%%%%%%%%%%%%%%%%%%%%%%%%%%%%%%%%%%%%%%%%%%%%%%%%%%%%%%%%%%%%%%%%%%%%
\begin{frame}
    \begin{theorem} \label{4.2.8}
        Let $ L $ be a lattice and $ u \in L $. Then the following conditions are equivalent. 
        \begin{enumerate}[(i)]
            \item
                There exist a decomposition $ x \oplus y = 1 $ of $ L $ such that $ x \leq u, y \wedge u \ll y/0 $
            \item
                There exist a direct summand $ x $ of $ L $ and $ y \ll L $ such that $ x \leq u $, $ u = x \vee y $.
            \item
                There exist a direct summand $ x $ of $ L $ such that $ u $ lie above $ x $
            \item
                There exist a supplement $ k $ of $ u $ in $ L $ such that $ u \wedge k $ a direct summand of $ u/0 $. 
        \end{enumerate}
    \end{theorem}
\end{frame}


%%%%%%%%%%%%%%%%%%%%%%%%%%%%%%%%%%%%%%%%%%%%%%%%%%%%%%%%%%%%%%%%%%%%%
%%%%%%%%%%%%%%%%%%%%%%%%%%%%%%%%%%%%%%%%%%%%%%%%%%%%%%%%%%%%%%%%%%%%%
%%%%%%%%%%%%%%%%%%%%%%%%%%%%%%%%%%%%%%%%%%%%%%%%%%%%%%%%%%%%%%%%%%%%%
\begin{frame}
    \setbeamercovered{invisible}
    We show that some properties of amply supplemented lattices in following theorems. 
    \pause
    \begin{lemma} \label{4.2.9}
        Let $ L $ be a amply supplemented lattice. Then for each supplement $ v $ in $ L $, sublattice $ v/0 $ of $ L $ is also amply supplemented.
    \end{lemma}
\end{frame}


%%%%%%%%%%%%%%%%%%%%%%%%%%%%%%%%%%%%%%%%%%%%%%%%%%%%%%%%%%%%%%%%%%%%%
%%%%%%%%%%%%%%%%%%%%%%%%%%%%%%%%%%%%%%%%%%%%%%%%%%%%%%%%%%%%%%%%%%%%%
%%%%%%%%%%%%%%%%%%%%%%%%%%%%%%%%%%%%%%%%%%%%%%%%%%%%%%%%%%%%%%%%%%%%%
\begin{frame}
    \begin{theorem} \label{4.2.10}
        Let $ L $ be a lattice. Then following conditions are equivalent.
        \begin{enumerate}[(i)]
            \item
                $ L $ is amply supplemented.
            \item
                For every $ u \in L $, there exist $ x/0 $ supplemented sublattice of $ L $ and $ y \ll L $ such that $ u = x \vee y $.
            \item
                For every $ u \in L $, there exist $ x/0 $ supplemented sublattice of $ L $ such that $ u $ lies above $ x $.
        \end{enumerate}
    \end{theorem}
\end{frame}



%%%%%%%%%%%%%%%%%%%%%%%%%%%%%%%%%%%%%%%%%%%%%%%%%%%%%%%%%%%%%%%%%%%%%
%%%%%%%%%%%%%%%%%%%%%%%%%%%%%%%%%%%%%%%%%%%%%%%%%%%%%%%%%%%%%%%%%%%%%
%%%%%%%%%%%%%%%%%%%%%%%%%%%%%%%%%%%%%%%%%%%%%%%%%%%%%%%%%%%%%%%%%%%%%
\begin{frame}
    \begin{theorem} \label{4.2.11}
    Let $ L $ be a lattice. Then following conditions are equivalent.
        \begin{enumerate}[(i)]
            \item
                $ L $ is amply supplemented and every supplement element in $ L $ is also direct summand of $ L $.
            \item
                Every element of $ L $ is lies above a direct summand of $ L $.
        \end{enumerate}
    \end{theorem}
\end{frame}


%%%%%%%%%%%%%%%%%%%%%%%%%%%%%%%%%%%%%%%%%%%%%%%%%%%%%%%%%%%%%%%%%%%%%
%%%%%%%%%%%%%%%%%%%%%%%%%%%%%%%%%%%%%%%%%%%%%%%%%%%%%%%%%%%%%%%%%%%%%
%%%%%%%%%%%%%%%%%%%%%%%%%%%%%%%%%%%%%%%%%%%%%%%%%%%%%%%%%%%%%%%%%%%%%
\begin{frame}
    \begin{theorem} \label{4.2.12}
    Let $ L $ be a lattice. Consider the following conditions:
     \begin{enumerate}[(i)]
            \item
                $ L $ is $ lifting $.
            \item
                $ L $ is $ H $-$ supplemented $.
            \item
                $ L $ is $ G_* $-$ lifting $.
            \item
                $ L $ is $ G_* $-$ supplemented $.
            \item
                $ L $ is $ supplemented $.
        \end{enumerate}
        Then, (i) $ \Rightarrow $ (ii) $ \Leftrightarrow $ (iii) $ \Rightarrow $ (iv) $ \Rightarrow $ (v).
    \end{theorem}
\end{frame}


%%%%%%%%%%%%%%%%%%%%%%%%%%%%%%%%%%%%%%%%%%%%%%%%%%%%%%%%%%%%%%%%%%%%%
%%%%%%%%%%%%%%%%%%%%%%%%%%%%%%%%%%%%%%%%%%%%%%%%%%%%%%%%%%%%%%%%%%%%%
%%%%%%%%%%%%%%%%%%%%%%%%%%%%%%%%%%%%%%%%%%%%%%%%%%%%%%%%%%%%%%%%%%%%%
\begin{frame}
    \begin{corollary} \label{4.2.13}
    Let $ L $ be a lattice. Then following conditions are true
        \begin{enumerate}[(i)]
            \item
                $ L $ is lifting if and only if $ L $ is amply supplemented and $ L $ is strongly $ \oplus $-supplemented.
            \item
                If $ L $ is $ G_* $-lifting, then $ G_* $-supplemented and $ \oplus $-supplemented.
            \item
                If $ L $ is $ G_* $-supplemented and strongly $ \oplus $-supplemented, then $ G_* $-lifting.
        \end{enumerate}
    \end{corollary}
\end{frame}
            
     
                
        
    
    

                
        







% Placing a * after \section means it will not show in the
% outline or table of contents.


% All of the following is optional and typically not needed. 


\end{document}


